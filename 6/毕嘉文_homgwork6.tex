\documentclass[UTF8]{ctexart}
\usepackage{graphicx,amsmath,geometry,tikz}
\usepackage{lmodern}
\geometry{a4paper,scale = 0.7}

\usepackage{gnuplottex}

\usepackage{xcolor}

\usetikzlibrary{positioning, shapes.geometric}
\title{作业六: 一元二次方程在实数域上的求解}
\author{毕嘉文 \\ 数学与应用数学\ 3190105194}
\date{\today}

\begin{document}
\maketitle

\section{数学原理}
对于实数域上的一元二次方程$ax^2 + bx + c = 0$,我们可以知道方程在虚数域上的解为$$
x_{1,2} = \dfrac{-b \pm \sqrt{b^2 - 4ac}}{2a}
$$
因此我们只要讨论$b^2 - 4ac$ 与$0$的关系
\section{算法流程图}
\begin{tikzpicture}[node distance=10pt]
	\node[draw, rounded corners]	(start)	{Start};
	\node[draw, diamond, aspect=2, below = of start]	(choice)	{a = 0};
	\node[draw, below = 30pt of choice]	(choice 1)	{b=0};
	\node[draw, right = 30pt of choice 1]	(delta)	{$b^2 - 4ac$};
	\node[draw, below = 30pt of delta] (delta2) {$x_{1,2} = \dfrac{-b \pm \sqrt{b^2 - 4ac}}{2a}$};
	\node[draw, right = 20pt of delta2] (delta1) {$x = -\frac{b}{2a} $};
	\node[draw, right = 20pt of delta1] (delta3) {无解};
	\node[draw, left = 30pt of choice 1] (out 1)	{$x = -\frac{c}{b}$};
	\node[draw, left = 30pt of  delta2]	(choice 2)	{c=0};
	\node[draw, left = 30pt of choice 2] (out 2)	{无解};
	\node[draw, below = 30pt of choice 2] (out 3)	{任意解};
	
	\draw[->] (start)  -- (choice);
	\draw[->] (choice) -- node[left]  {Yes} (choice 1);
	\draw[->] (choice) -- node[above] {No}  (delta);
	\draw[->] (choice	1) -- node[above] {No}  (out 1);
	\draw[->] (choice	1) -- node[left]  {Yes} (choice 2);
	\draw[->] (choice	2) -- node[above] {No}  (out 2);
	\draw[->] (choice	2) -- node[left] {Yes}  (out 3);
	\draw[->] (delta) -- node[left] {$>0$} (delta2);
	\draw[->] (delta) -- node[left] {$=0$} (delta1);
	\draw[->] (delta) -- node[above] {$<0$} (delta3);
\end{tikzpicture}
\section{示意图}
以下为一元二次方程(默认$a\neq 0$)的三种情况的示意图,在这里以$y=x^2$,$y=x^2+10$,$y=x^2-10$为例。
\begin{figure}[h]
\centering%
\begin{gnuplot}[terminal=epslatex,terminaloptions={color size 14cm,12cm}]
	set key top left
	set xlabel 'x [1]'
	set ylabel 'y [1]'
	plot x**2 ,x**2+10 , x**2-10 
\end{gnuplot}
\caption{$y=x^2$,$y=x^2+10$,$y=x^2-10$示意图}%
\label{pic:cairolatex}%
\end{figure}%


\bibliographystyle{plain}
\bibliography{毕嘉文_homework6}
\end{document}
