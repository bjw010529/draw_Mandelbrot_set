\documentclass[UTF8]{ctexart}
\usepackage{graphicx,amsmath,geometry}
\usepackage[colorlinks,linkcolor=blue]{hyperref}
\geometry{a4paper,scale = 0.7}

\title{作业三: 关于我的Linux环境}


\author{毕嘉文 \\ 数学与应用数学\ 3190105194}

\begin{document}
\maketitle

\section{Linux介绍}
我使用的Linux环境是WSL2 Ubuntu,使用的Ubuntu版本为\verb|Ubuntu 18.04.6 LTS|,使用\verb|lsb_release -a|指令
可以获取版本信息\begin{verbatim}
No LSB modules are available.
Distributor ID: Ubuntu
Description:    Ubuntu 18.04.6 LTS
Release:        18.04
Codename:       bionic
\end{verbatim}

\section{个性化改动}
从大一开始的时候,我就开始使用WSL + Ubuntu,当时主要是使用Windows下的VS Code作为文本编辑器,使用VS Code下的Terminal来调试和运行,主要任务是
使用latex写些文章,参考了如下文章\href{https://zhuanlan.zhihu.com/p/65931654}{Ubuntu 使用 Latex,使用VS Code 中文}。
经过一段时间的适应之后,我把C++的工作环境从Dev C++迁移到WSL Ubuntu + VS Code上。然后在一段时间的使用之后,我感觉VS Code插件用来编译和运行很麻烦,
也经常会出bug,就在朋友推荐下使用了Sublime Text3作为文本编辑器,直接用Ubuntu Terminal来编译和运行。\par
我又针对不同的应用场景安装了各种软件包,比如为了数学实践大作业安装了doxygen,cmake,为了实习工作安装了线性代数计算库armadillo\cite{armadillo},
为了学习AI、训练AI模型安装了jupyter notebook 和框架pytorch,为了远程编辑安装了vim。
另外,我在Windows上安装了自带的OpenSSH,在iPad上安装了Terminus,现在可以用iPad通过ssh访问电脑终端,当然也就可以远程使用WSL。

\section{未来工作}
\subsection{未来半年使用场景}
未来半年我在这套系统上的主要任务大概就是写论文吧,工作量不算大,要用到的软件也不算多。
\subsection{未来需求}
这套工作环境应该不会符合未来工作需求,因为我在配置的时候主要是为了方便,所以选用的工具尽可能轻量级,
包括使用轻量级的文本编辑器vim和Sublime Text,使用Linux的替代品WSL,可能不会满足未来工作需求。
未来如果要处理科研/学术任务,可能会考虑emacs + VirtualBox Ubuntu的环境;如果要处理商业化的工作,可能会使用WSL + CLion/VS Code的环境。


\section{如何保证环境稳定}
从大二开始我就将每学期保存下来的文件打包压缩存在百度云,之前也尝试过将数学实践大作业保存在github中
(参考\href{https://github.com/bjw010529/test-work}{这里})。未来考虑会购买云盘服务,习惯于使用git提交代码。


\bibliographystyle{plain}
\bibliography{毕嘉文_homework3}
\end{document}
